\documentclass[12pt, a4paper]{article}

\usepackage[left=0.5cm,top=0cm,right=0.5cm]{geometry}
\usepackage{graphicx}
\usepackage{amsmath}
\usepackage{minted}

\def\grosso{\vspace{-1.3em}\begin{center}\rule{200mm}{4pt}\end{center}}
\def\fino{\vspace{-1.5em}\begin{center}\rule{200mm}{0.5pt}\end{center}}

\date{}
\author{}
\title{
    \includegraphics[width=4cm]{logo.png} \\
    \vspace{-0.5cm} \line(1,0){400} \vspace{0.5cm} \\
    \Large \textbf{SCC0220 - Laboratório Introdução à Ciência da Computação II} \\
    \vspace{1em}
    \large \textbf{Relatório de execução da aula prática 1} \\
    \vspace{-3.5em}
}

\begin{document}

\maketitle

\begin{tabular}{ll}
    \textbf{Alunos}      & \textbf{NUSP} \\
    Murilo Louco Louco   & 11111111 \\
    Outra Pessoa Maneira & 11111112 \\
\end{tabular}
\vspace{1.5em}
\sffamily

\large \textbf{Exercício 1 - Exercício maneiro}
\grosso

\large  \textbf{Item a}
\fino

\large $\rightarrow$ \textbf{Comentário} \small

O programa "Hello World" é o exemplo mais simples em C, utilizado para introduzir a sintaxe básica, mostrando como imprimir texto no console e servir de primeiro passo para iniciantes em programação.

\large $\rightarrow$ \textbf{Código} \small

\begin{minted}[fontsize=\small]{c}
    #include <stdio.h>

    int main() {
        puts("Hello World!");
        return 0;
    }
\end{minted}

\large $\rightarrow$ \textbf{Saída} \small

A saída esperada do programa é:

\begin{verbatim}
    Hello World!
\end{verbatim}

\end{document}
